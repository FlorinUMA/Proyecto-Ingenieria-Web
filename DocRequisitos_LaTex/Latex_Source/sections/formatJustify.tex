Hemos decidido utilizar dos formatos, uno tabular para los requisitos funcionales y uno más simplista para los requisitos no funcionales. Esto se ha hecho pensando en el modelo de trabajo que se utilizará durante el desarrollo de la web, ya que con total seguridad se aplicará la metodología SCRUM. Además al inicio podremos observar una tabla con la prioridad de cada requisito, que podrá ser Prioritario, Necesario o Deseable.

\subsection{Formato de requisitos funcionales}
Para los requisitos funcionales hemos decidido utilizar un modelo de tabla basado en User Stories, es así para que sea más comprensible para un cliente que no esté relacionado con el ámbito informático. En el se recogen la identificación del requisito (mediante el código RF-n, siendo n el número del requisito), el usuario que utilizará dicha característica, el requisito en cuestión, el porqué del requisito, los criterios de aceptación de posee el requisito y la versión del mismo. En la tabla \ref{tab:formatoRF} podemos observar el modelo de tabla que se va a utilzar para los requisitos funcionales.

\newcolumntype{C}[1]{>{\centering\arraybackslash}p{#1}}
\begin{table}[H]
    \label{tab:formatoRF}
 	\caption{Modelo de requisito funcional}
	\centering
	
	\begin{tabular}{C{3cm}|C{12cm}}
 		\toprule
 		\textbf{Atributo} & \textbf{Descripción} \\
 		\midrule
 	    ID & RF-n \\
 	    Cómo... & <Usuario> \\
 	    Quiero... & <Funcionalidad> \\
 	    Para obtener... & <Objetivo de la funcionalidad>  \\
 	    Criterio de Aceptación & <Criterio> \\
 	    Versión & <número de versión> \\
 		\bottomrule
 		\end{tabular}
\end{table}

\subsection{Formato de requisitos no funcionales}
Los requisitos no funcionales, debido a su menor descripción y falta de información actualmente, se han decido exponer en formato lista. Enumerándolos de mayor a menor importancia en el desarrollo.