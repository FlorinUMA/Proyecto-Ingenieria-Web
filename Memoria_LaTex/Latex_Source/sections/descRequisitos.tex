\subsection{Requisitos funcionales}
A continuación, observamos en la tabla \ref{tab:prioridadRequisitosFuncionales} la prioridad de los requisitos funcionales.
% Florin -> DONE
\newcolumntype{C}[1]{>{\centering\arraybackslash}p{#1}}
\begin{table} [H]
    \label{tab:prioridadRequisitosFuncionales}
 	\caption{Requisitos funcionales}
	\centering
	
	\begin{tabular}{C{1cm}|C{11cm}|C{3cm}}
 		\toprule
 		\textbf{ID} & \textbf{Descripción} & \textbf{Prioridad} \\
 		\midrule
 		RF-1 & Posibilidad de cancelar la ejecución de una tarea del robot & Necesario \\
   		RF-2 & Poder asignar varias tareas a un robot & Deseable \\
   		RF-3 & Asignar varios robots para una misma tarea & Deseable\\
   		RF-4 & Los robots podrán notificar fallos durante la realización de una tarea & Prioritario\\
   		RF-5 & Poder tener un filtro en el que seleccionar por tipos los robots que se muestran en la vista & Deseable\\
   		RF-6 & Implementar tareas básicas para el robot y que sea escalable desde la interfaz. & Deseable\\
   		RF-7 & Visualizar el estado de cada uno de los robots. & Necesario\\
   		RF-8 & Guardar el historial de cada robot automáticamente. & Necesario\\
   		RF-9 & En la monitorización del robot se deberá poder acceder a todos los tipos de tareas & Necesario\\
   		RF-10 & Autorización mediante usario-contraseña para entrar en la aplicación . & Deseable\\
   		RF-11 & Los robots notifican a la aplicación si estos no pueden realizar la tarea por alguna razón. & Prioritario\\
 		\bottomrule
 		\end{tabular}

\end{table}


\begin{table}[H]
    \label{tab:reqF1}
 	\caption{Descripción requisito RF-1}
	\centering
	
	\begin{tabular}{C{3cm}|C{12cm}}
 		\toprule
 		\textbf{Atributo} & \textbf{Descripción} \\
 		\midrule
 	    ID & RF-1 \\
 	    Cómo... & Empleado sanitario \\
 	    Quiero... & Poder cancelar en todo momento una tarea asignada a un robot determinado. \\
 	    Para obtener... & Mayor control sobre la actividad del robot. \\
 	    Criterio de Aceptación & Poder cancelar varias tareas sin que haya errores inesperados. \\
 	    Versión & 1.1 \\
 		\bottomrule
 		\end{tabular}
\end{table}

\begin{table}[H]
    \label{tab:reqF2}
 	\caption{Descripción requisito RF-2}
	\centering
	
	\begin{tabular}{C{3cm}|C{12cm}}
 		\toprule
 		\textbf{Atributo} & \textbf{Descripción} \\
 		\midrule
 	    ID & RF-2 \\
 	    Cómo... & Empleado sanitario \\
 	    Quiero... & Poder asignar varias tareas en secuencia para el robot. \\
 	    Para obtener... & La posibilidad de asignar al robot una lista de tareas que la complete en un orden concreto.  \\
 	    Criterio de Aceptación & Poder asignar varias tareas sucesivas al robot y que este las ejecute en el orden que se dieron. \\
 	    Versión & 1 \\
 		\bottomrule
 		\end{tabular}
\end{table}

\begin{table}[H]
    \label{tab:reqF3}
 	\caption{Descripción requisito RF-3}
	\centering
	
	\begin{tabular}{C{3cm}|C{12cm}}
 		\toprule
 		\textbf{Atributo} & \textbf{Descripción} \\
 		\midrule
 	    ID & F3 \\
 	    Cómo... & Empleado sanitario \\
 	    Quiero... & Poder asignar a más de un robot para realizar la misma tarea. \\
 	    Para obtener... & La capacidad de reducir el tiempo necesario de una tarea ya que la harán dos robots a la vez.  \\
 	    Criterio de Aceptación & Poder asignar a una misma tarea tres robots o menos. \\
 	    Versión & 1 \\
 		\bottomrule
 		\end{tabular}
\end{table}

\begin{table}[H]
    \label{tab:reqF4}
 	\caption{Descripción requisito RF-4}
	\centering
	
	\begin{tabular}{C{3cm}|C{12cm}}
 		\toprule
 		\textbf{Atributo} & \textbf{Descripción} \\
 		\midrule
 	    ID & RF-4 \\
 	    Cómo... & Empleado sanitario \\
 	    Quiero... & Poder ver si ha ocurrido un problema durante el desarrollo de la tarea del robot.\\
 	    Para obtener... & Una información más detallada sobre cada trabajo que hayan llevado a cabo el robot.  \\
 	    Criterio de Aceptación & Poder visualizar el estado de fallo correctamente, siempre que se haya dado dicho problema. \\
 	    Versión & 1 \\
 		\bottomrule
 		\end{tabular}
\end{table}


% Claudia. Revisado por Florin
%Sería deseable poder tener un filtro en el que seleccionar por tipos los robots que se muestran en la vista. (CAMBIAR A REQUISITO FUNCIONAL)

\begin{table}[H]
    \label{tab:reqF5}
 	\caption{Descripción requisito RF-5}
	\centering
	
	\begin{tabular}{C{3cm}|C{12cm}}
 		\toprule
 		\textbf{Atributo} & \textbf{Descripción} \\
 		\midrule
 	    ID & RF-5 \\
 	    Cómo... &  Técnico Sanitario\\
 	    Quiero... &  Implementar un filtro en la aplicación. \\
 	    Para obtener... &  Seleccionar por tipos los robots que se muestran en la vista. \\
 	    Criterio de Aceptación &  Poder seleccionar y aplicar un único filtro (simultáneamente)\\
 	    Versión & 1.1 \\
 		\bottomrule
 		\end{tabular}
\end{table}



\begin{table}[H]
    \label{tab:reqF6}
 	\caption{Descripción requisito RF-6}
	\centering
	
	\begin{tabular}{C{3cm}|C{12cm}}
 		\toprule
 		\textbf{Atributo} & \textbf{Descripción} \\
 		\midrule
 	    ID & RF-6 \\
 	    Cómo... & Técnico Sanitario \\
 	    Quiero... & Implementar tareas básicas para el robot. \\
 	    Para obtener... & Asignarles a los robots una o más tareas básicas.  \\
 	    Criterio de Aceptación & El objetivo es que así el sistema sea escalable a más tareas. \\
 	    Versión & 1 \\
 		\bottomrule
 		\end{tabular}
\end{table}

\begin{table}[H]
    \label{tab:reqF7}
 	\caption{Descripción requisito RF-7}
	\centering
	
	\begin{tabular}{C{3cm}|C{12cm}}
 		\toprule
 		\textbf{Atributo} & \textbf{Descripción} \\
 		\midrule
 	    ID & RF-7 \\
 	    Cómo... & Empleado Sanitario \\
 	    Quiero... & Visualizar el estado de cada uno de los robots. \\
 	    Para obtener... & Información acerca de cuál está libre.  \\
 	    Criterio de Aceptación & Que sea intuitivo y fácil de utilizar para el usuario medio. \\
 	    Versión & 1 \\
 		\bottomrule
 		\end{tabular}
\end{table}

% Juan Carlos

\begin{table}[H]
    \label{tab:reqF8}
 	\caption{Descripción requisito RF-8}
	\centering
	
	\begin{tabular}{C{3cm}|C{12cm}}
 		\toprule
 		\textbf{Atributo} & \textbf{Descripción} \\
 		\midrule
 	    ID & RF-8 \\
 	    Cómo... & Técnico Sanitario \\
 	    Quiero... & Guardar el historial de cada robot automáticamente. \\
 	    Para obtener... & Información acerca de los eventos pasados del robot.  \\
 	    Criterio de Aceptación & Que se guarden los datos de forma diferenciada e inequívoca  \\
 	    Versión & 1 \\
 		\bottomrule
 		\end{tabular}
\end{table}
\begin{table}[H]
    \label{tab:reqF9}
 	\caption{Descripción requisito RF-9}
	\centering

	\begin{tabular}{C{3cm}|C{12cm}}
 		\toprule
 		\textbf{Atributo} & \textbf{Descripción} \\
 		\midrule
 	    ID & RF-9 \\
 	    Cómo... & Técnico sanitario \\
 	    Quiero... & Poder ver y gestionar en la monitorización del robot los tipos de tareas que este pueda desempeñar. \\
 	    Para obtener... & Una visión general de las diferentes tareas que puede desempeñar un tipo de robot y poder modificarlas en caso de que sea necesario  \\
 	    Criterio de Aceptación & Obtener una vista de todas las tareas posibles de un robot y poder manipularlas de manera sencilla \\
 	    Versión & 2\\
 		\bottomrule
 		\end{tabular}
\end{table}

\begin{table}[H]
    \label{tab:reqF10}
 	\caption{Descripción requisito RF-10}
	\centering

	\begin{tabular}{C{3cm}|C{12cm}}
 		\toprule
 		\textbf{Atributo} & \textbf{Descripción} \\
 		\midrule
 	    ID & RF-10 \\
 	    Cómo... & Técnico sanitario \\
 	    Quiero... & Que los usuarios de la aplicación puedan entrar a la aplicación con un inicio de sesión básico (ID, contraseña y rol) \\
 	    Para obtener... & Mayor seguridad en el acceso a la aplicación  \\
 	    Criterio de Aceptación & Que la aplicación no sea accesible por usuarios sin permisos \\
 	    Versión & 1 \\
 		\bottomrule
 		\end{tabular}
\end{table}

\begin{table}[H]
    \label{tab:reqF11}
 	\caption{Descripción requisito RF-11}
	\centering

	\begin{tabular}{C{3cm}|C{12cm}}
 		\toprule
 		\textbf{Atributo} & \textbf{Descripción} \\
 		\midrule
 	    ID & RF-11 \\
 	    Cómo... & Empleado sanitario \\
 	    Quiero... & Que los robots notifiquen a la aplicación que no pueden realizar la tarea por alguna razón \\
 	    Para obtener... & Información necesaria y rápida sobre los robots en ese estado  \\
 	    Criterio de Aceptación & Que la aplicación reciba el aviso de que el robot no ha podido ejecutar la tarea enmendada \\
 	    Versión & 1 \\
 		\bottomrule
 		\end{tabular}
\end{table}

\subsection{Requisitos no funcionales}
Observamos en la tabla \ref{tab:prioridadRequisitosNoFuncionales} la prioridad de los requisitos no funcionales, así como su tipo.
\begin{table} [H]
    \label{tab:prioridadRequisitosNoFuncionales}
 	\caption{Requisitos no funcionales}
	\centering
	
	\begin{tabular}{C{1cm}|C{8cm}|C{3cm}|C{3cm}}
 		\toprule
 		\textbf{ID} & \textbf{Descripción} & \textbf{Tipo} & \textbf{Prioridad} \\
 		\midrule
 		RNF1 & Se deberá proporcionar un sistema con interfaz de usuario & Interfaz de usuario & Prioritario \\
 		RNF2 & El programa debe tener una naturaleza sencilla para poder configurar o manejar a distintos tipos de robots & Usabilidad & Necesario \\
 		RNF3 & Se debe poder ver lo que estan haciendo los robots de un vistazo & Usabilidad & Prioritario \\
 		RNF4 & La aplicación debe soportar al menos dos usuarios simultáneamente & Mantenibilidad & Necesario \\
 		RNF5 & Deberán existir distintos tipos de error para el robot & Fiabilidad & Deseable \\
 		RNF6 & La información asociada a una tarea deberá incluir la fecha y la hora a la que se ha realizado & Usabilidad & Deseable \\
 		RNF7 & Solo se almacenará el historial de la última jornada & Usabilidad & Deseable \\
 		RNF8 & Recibir información de los estados de los robots con una asincronía & Interfaz de Usuario & Deseable \\
   	
 		\bottomrule
 		\end{tabular}

\end{table}
\subsubsection{Requisitos de Interfaz de Usuario}

\begin{table}[H]
    \label{tab:reqNF1}
 	\caption{Descripción requisito RNF-1}
	\centering
	\begin{tabular}{C{3cm}|C{12cm}}
 		\toprule
 		\textbf{Atributo} & \textbf{Descripción} \\
 		\midrule
 	    ID & RNF-1 \\
 	    Descripción & Se deberá proporcionar un sistema con interfaz de usuario \\
 	    Precedente & RF-7 \\
 	    Versión & 1 \\
 		\bottomrule
 		\end{tabular}
\end{table}

\begin{table}[H]
    \label{tab:reqNF9}
 	\caption{Descripción requisito RNF-8}
	\centering

	\begin{tabular}{C{3cm}|C{12cm}}
 		\toprule
 		\textbf{Atributo} & \textbf{Descripción} \\
 		\midrule
 	    ID & RNF-8 \\
 	    Descripción & Recibir información de los estados de los robots con una asincronía máxima de 30 segundos \\
 	    Precedente & RF-7 \\
 	    Versión & 1 \\
 		\bottomrule
 		\end{tabular}
\end{table}


\subsubsection{Requisitos de Usabilidad}

\begin{table}[H]
    \label{tab:reqNF1}
 	\caption{Descripción requisito RNF-2}
	\centering

	\begin{tabular}{C{3cm}|C{12cm}}
 		\toprule
 		\textbf{Atributo} & \textbf{Descripción} \\
 		\midrule
 	    ID & RNF-2 \\
 	    Descripción & El programa debe tener una naturaleza sencilla para poder configurar o manejar a distintos tipos de robots\\
 	    Precedente & RF-2 \\
 	    Versión & 1 \\
 		\bottomrule
 		\end{tabular}
\end{table}
\begin{table}[H]
    \label{tab:reqNF1}
 	\caption{Descripción requisito RNF-3}
	\centering

	\begin{tabular}{C{3cm}|C{12cm}}
 		\toprule
 		\textbf{Atributo} & \textbf{Descripción} \\
 		\midrule
 	    ID & RNF-3 \\
 	    Descripción & Se debe poder ver lo que están haciendo los robots de un vistazo \\
 	    Precedente & X \\
 	    Versión & 1 \\
 		\bottomrule
 		\end{tabular}
\end{table}

\begin{table}[H]
    \label{tab:reqNF7}
 	\caption{Descripción requisito RNF-6}
	\centering

	\begin{tabular}{C{3cm}|C{12cm}}
 		\toprule
 		\textbf{Atributo} & \textbf{Descripción} \\
 		\midrule
 	    ID & RNF-6 \\
 	    Descripción & La información asociada a una tarea deberá incluir la fecha y la hora a la que se ha realizado. \\
 	    Precedente & RF-8 \\
 	    Versión & 1 \\
 		\bottomrule
 		\end{tabular}
\end{table}

\begin{table}[H]
    \label{tab:reqNF8}
 	\caption{Descripción requisito RNF-7}
	\centering

	\begin{tabular}{C{3cm}|C{12cm}}
 		\toprule
 		\textbf{Atributo} & \textbf{Descripción} \\
 		\midrule
 	    ID & RNF-7 \\
 	    Descripción & Solo se almacenará el historial de la última jornada. \\
 	    Precedente & RF-8 \\
 	    Versión & 1 \\
 		\bottomrule
 		\end{tabular}
\end{table}


\subsubsection{Requisitos de Mantenibilidad}

\begin{table}[H]
    \label{tab:reqNF1}
 	\caption{Descripción requisito RNF-4}
	\centering

	\begin{tabular}{C{3cm}|C{12cm}}
 		\toprule
 		\textbf{Atributo} & \textbf{Descripción} \\
 		\midrule
 	    ID & RNF-4 \\
 	    Descripción & La aplicación debe soportar al menos dos usuarios simultáneamente \\
 	    Precedente & X \\
 	    Versión & 1 \\
 		\bottomrule
 		\end{tabular}
\end{table}

\subsubsection{Requisitos de Fiabilidad}

% Revisado por Florin
\begin{table}[H]
    \label{tab:reqNF1}
 	\caption{Descripción requisito RNF-5}
	\centering

	\begin{tabular}{C{3cm}|C{12cm}}
 		\toprule
 		\textbf{Atributo} & \textbf{Descripción} \\
 		\midrule
 	    ID & RNF-5 \\
 	    Descripción & Se deberá mostrar los mensajes de error que se hayan producido de forma amigable con el usuario \\
 	    Precedente & RF-4 \\
 	    Versión & 1.1 \\
 		\bottomrule
 		\end{tabular}
\end{table}

% -----------------


