En esta sección vamos a realizar un informe sobre la usabilidad de nuestra aplicación apoyándonos en los principios de usabilidad de Nielsen.

Para ello iremos vista por vista en nuestra aplicación, después de hacer un resumen general sobre principios que se cumplan en todas y cada una de las vistas para así acortar el informe.
\subsection{Usabilidad general}
En la aplicación, podemos ver con facilidad que se cumplen varios principios de usabilidad de Nielsen.

Uno de ellos es \textit{Consistencia y estándares}, ya que la totalidad de la aplicación tiene el mismo formato: fondo gris, tablas de fondo negro y letras blancas. También, todos los botones son con el mismo formato para botones de acción (azules) y los de cancelación (rojos). 
Cuando se pulsa en el logo de la aplicación, este nos envía a la vista base. Además, en todas las vistas vemos tenemos la barra superior para realizar dicha acción, a la vez que la inferior para poder ver información sobre los participantes del proyecto.

Otro principio de Nielsen que se cumple es el de \textit{Diálogos estéticos y de diseño minimalista}. Este principio lo cumple, ya que las vistas de la aplicación no contienen información innecesaria que nos pueda entorpecer la comprensión de la
información relevante al no existir ruido de información que puede provocar esta
información extra. Además de tener un diseño minimalista, ya que las páginas no tienen
gran número de colores, pocos elementos y mucho espacio, incluso en la vista base, la cual tiene texto para facilitar la comprensión de la aplicación.

Otro principio de Nielsen sería el de \textit{Ayudar a los usuarios a reconocer, diagnosticar y recuperarse de los errores}. Cuando
surge un error, la aplicación envía una vista de error instando al usuario a dirigirse a la vista base.

\subsection{Login}
\subsection{Vistas de Médico}
\subsubsection{Listado de tareas}
\subsubsection{Asignación de tareas}

\subsection{Vistas de Técnico}
\subsubsection{Listado de robots}
\subsubsection{Detalles de los robots}
\subsubsection{Creación de tareas}

